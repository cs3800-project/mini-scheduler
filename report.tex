%
%You can keep the 12pt font size, or go to 11pt or (default) 10pt
% Do NOT go any larger than 12pt font size for submission
%
%If you want to edit a printed copy, you may want to add draft mode
% (as in \documentclass[draft,conference,12pt]{IEEEtran})
% This adds space between the lines providing easier editing markup
%
%For more details see
% http://ras.papercept.net/conferences/support/files/IEEEtran_HOWTO.pdf
%
\documentclass[conference,11pt]{IEEEtran}
\usepackage{cite}
\usepackage{graphicx}
\usepackage{algorithmic}
\usepackage{url}
\usepackage{flushend}

% correct bad hyphenation here
\hyphenation{op-tical net-works semi-conduc-tor}

\begin{document}

%
% paper title
% Titles are generally capitalized except for words such as a, an, and, as,
% at, but, by, for, in, nor, of, on, or, the, to and up, which are usually
% not capitalized unless they are the first or last word of the title.
% Linebreaks \\ can be used within to get better formatting as desired.
% Do not put math or special symbols in the title.
\title{Optimal Scheduling Algorithms for Varying Process Sets}

\author{\IEEEauthorblockN{Rachael Engle, Anna Pankiewicz, and Albert Perlman}
\IEEEauthorblockA{Computer Science Department\\
Missouri University of Science and Technology\\
Rolla, MO 65409}}

% make the title area
\maketitle

\begin{abstract}
This work explores the advantages and disadvantages of using particular scheduling algorithms with different types of processes. The five specific algorithms analyzed were First In First Out, Round Robin, Shortest Job Next, Shortest Remaining Time, and Highest Response Ratio Next.
\end{abstract}

\section{Background}
Through the operating systems class this semester, each of these five scheduling algorithms were discussed. A comparison was also provided, highlighting differences in throughput, response time, overhead, effect on processes, and starvation.

While these given comparisons are fantastic, the purpose of our project was to implement these scheduling algorithms and observe the differences ourselves. As we are only simulating one processor, some metrics are difficult to observe, such as throughput. However, by analyzing other metrics such as turnaround time, normalized turnaround time, and response time, we can gain insight into the more qualitative performance metrics.

\section{Introduction}
Our program design centered around a simulated system capable of handling incoming jobs, scheduling them, and allowing the user to select a scheduling function. A simulated job file is provided at the command line, and this file contains the process names, arrival times, and execution times. For simplicity's sake, the execution time of a process is provided, as it would be outside the scope of the project to analyze previous jobs and their characteristics to provide even remotely accurate time estimates.

This aforementioned simulated system is not alone in handling process arrivals and requests. Along with the "system", there is a scheduler that runs the selected algorithm until all incoming processes have been completed. Every time cycle, this scheduler checks for jobs arriving at the current time, adds them to the appropriate queue of the running scheduling algorithm, and provides a snapshot of the system state at that time. When all jobs are finished, the scheduler concludes and produces a report of the algorithm's performance on that process set.

\section{Algorithms}


\subsection{First In First Out}


\subsection{Round Robin}


\subsection{Shortest Job Next}


\subsection{Shortest Remaining Time}


\subsection{Highest Response Ratio Next}


\section{Results}

\end{document}
